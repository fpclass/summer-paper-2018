
%%% Question 5 - - - - - - - - - - - - - - - - - - - - - - - - - - - - - -
\question This question is about monads.
\begin{parts} 
\part[4] Define a suitable instance of the \texttt{Monad} type class for the \texttt{Either} type:

\texttt{data Either a b = Left a | Right b} \droppoints 
\begin{solution} \emph{Application.}
\begin{verbatim}
instance Monad (Either e) where 
    return x = Right x
    
    Left y  >>= f = Left y 
    Right x >>= f = f x
\end{verbatim}
\end{solution}
\part[5] Monads are a more expressive abstraction than Applicative Functors. That is, some functions that can be implemented in terms of \texttt{(>{}>=)} cannot be implemented with just \texttt{fmap} and \texttt{(<*>)}. Using a suitable example, explain the limitations of Applicative Functors. \droppoints
\begin{solution}
\emph{Bookwork+Comprehension.} Applicative effects cannot depend on the results of previous computations. Consider the example of looking up the grandmother of a person from a dictionary which maps people to their mothers. This dictionary can be represented as a value of type \texttt{[(String, String)]} and we can then use the \texttt{lookup~::~Eq a => a -> [(a,b)] -> Maybe b} function from the standard library to retrieve the name of the mother of a given person:
\begin{verbatim}
grandmother :: String -> [(String, String)] -> Maybe String
grandmother x dict = do 
    mother <- lookup x dict
    lookup mother dict
\end{verbatim}
If there is no mapping for a person's name to their mother, then \texttt{Nothing} is returned. Therefore, to look up a person's grandmother, we first need to look up their mother's name and then the name of their mother's mother. 
\end{solution}
\part[10] Prove that your instance of the \texttt{Monad} type class for the \texttt{Either} type obeys the monad laws. \droppoints
\begin{solution}
\emph{Application.} We can prove left identity through simple equational reasoning to rewrite one side of the equation to the other:
\begin{displaymath}
\begin{array}{cl}
\expr{\mathit{return}~x \bind f}
\hint{Definition of $\mathit{return}$}
\expr{\mathit{Right}~x \bind f}
\hint{Definition of $\bind$}
\lastexpr{f~x}
\end{array}
\end{displaymath}
The proof for right identity is by case analysis on $m$. In the case that $m = \mathit{Left~a}$ for all $a$:
\begin{displaymath}
\begin{array}{cl}
\expr{\mathit{Left~a} \bind \mathit{return}}
\hint{Definition of $\bind$}
\lastexpr{\mathit{Left~a}}
\end{array}
\end{displaymath}
In the case that $m = \mathit{Right}~b$ for all $b$:
\begin{displaymath}
\begin{array}{cl}
\expr{\mathit{Right}~b \bind \mathit{return}}
\hint{Definition of $\bind$}
\expr{\mathit{return}~b}
\hint{Definition of $\mathit{return}$}
\lastexpr{\mathit{Right}~b}
\end{array}
\end{displaymath}
The proof for associativity is also by case analysis on $m$. In the case we have that $m = \mathit{Left~a}$ for all $a$:
\begin{displaymath}
\begin{array}{cl}
\expr{(\mathit{Left~a} \bind f) \bind g}
\hint{Definition of $\bind$}
\expr{\mathit{Left~a} \bind g}
\hint{Definition of $\bind$}
\expr{\mathit{Left~a}}
\hint{Unapplying $\bind$}
\lastexpr{\mathit{Left~a} \bind (\lambda x \to f~x \bind g)}
\end{array}
\end{displaymath}
In the case that $m = \mathit{Right}~b$ for all $b$: \allowdisplaybreaks
\begin{displaymath}
\begin{array}{cl}
\expr{(\mathit{Right}~b \bind f) \bind g}
\hint{Definition of $\bind$}
\expr{f~b \bind g}
\hint{$\eta$-expansion}
\expr{(\lambda x \to f~x \bind g)~b}
\hint{Unapplying $\bind$}
\lastexpr{\mathit{Right}~b \bind (\lambda x \to f~x \bind g)}
\end{array}
\end{displaymath}
\end{solution}
\part[3] Define a function 
\begin{center}
	\texttt{foldrM~::~Monad m => (a -> b -> m b) -> b -> [a] -> m b}
\end{center}
which generalises \texttt{foldr} to functions with monadic effects. \droppoints
\begin{solution} \emph{Application.}
\begin{verbatim}
foldrM f z []     = return z 
foldrM f z (x:xs) = do 
     z' <- foldrM f z xs
     f x z'
\end{verbatim}
\end{solution}
\part[3] The following, monadic program is written using the do-notation which is just syntactic sugar for \texttt{>{}>=} and anonymous functions. Show what the program would look like without the do-notation: \droppoints 
\begin{verbatim}
supers :: [(String, String)]
supers = [ ("Monad", "Applicative")
         , ("Applicative", "Functor") ]

main :: IO ()
main = do x <- getLine
          case (do y <- lookup x supers
                   lookup y supers) of
             Nothing -> return ()
             Just z  -> do putStrLn z
                           main
\end{verbatim}
\begin{solution} \emph{Comprehension.}
\begin{verbatim}
supers :: [(String, String)]
supers = [ ("Monad", "Applicative")
         , ("Applicative", "Functor") ]
	
main :: IO ()
main = getLine >>= \x ->
	   case (lookup x supers >>= \y ->
	         lookup y supers) of
	       Nothing -> return ()
	       Just z  -> putStrLn z >> main
\end{verbatim}
\end{solution}
\end{parts}
